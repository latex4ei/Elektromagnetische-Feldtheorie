w\documentclass[]{article}

% Pakete laden
\usepackage[utf8]{inputenc}		% Zeichenkodierung: UTF-8 (für Umlaute)   
\usepackage[german]{babel}		% Deutsche Sprache
\usepackage{multicol}			% ermöglicht Seitenspalten  
\usepackage{booktabs}			% bessere Tabellenlinien
\usepackage{graphicx}			% Zum Bilder einfügen benötigt
\usepackage{pbox}				% Intelligent parbox: \pbox{maximum width}{blabalbalb \\ blabal}
\usepackage{scientific}			% Eigenes Paket
\begin{document}

\title{Zusammenfassung EMF}
\author{Martin Zellner}

\section*{1. Tutorübung}

Poynting-Vektor: 
\begin{equation}
	\vec s = \vec E \times \vec H
\end{equation}
\begin{equation}
	[\vec s] = \frac{J}{m^2 s} = \frac{W}{m^2} = \frac{N}{m s}
\end{equation}
\begin{equation}
	\int \vec E d \vec s = U
\end{equation}
Durchfultungsgesetz:
\begin{equation}
	\int \limits_{\partial A} \vec H d \vec s = J(A) = \iint \limits_{A} \vec j d \vec a
\end{equation}
für Spulen
\begin{equation}
\int \limits_{\partial A} \vec H  d s = N J
\end{equation}

Materialgesetze
\begin{eqnarray}
	\vec B = \underbrace{\mu_0 \mu_r}_{\mu} \vec H \\
	\vec D = \underbrace{\epsilon_r \epsilon_0}_{\epsilon} \vec E
\end{eqnarray}
lokales ohmsches Gesetz
\begin{eqnarray}
	\vec j = \sigma \vec E
\end{eqnarray}
Energiedichte 

\begin{eqnarray}
	w_{mag} = \frac 1 2 \vec H \vec B = \frac{1}{2\mu} \abs{\vec B}^2 \\
	w_{el} = \frac 1 2 \vec E \vec D = \frac \epsilon 2 \abs{\vec E}^2
\end{eqnarray}

Bilanz
\begin{eqnarray}
	\underbrace{\frac{\partial x}{\partial t}}_{\text{zeitliche Veränderung}} + \underbrace{\div \vec J_x (\vec r, t)}_{\text{zu- oder abfluss}} = \underbrace{\pi}_{\text{Produktionsrate}} 
\end{eqnarray}


\section*{2. Tutorübung}
MWG: 
\begin{eqnarray}
\div \vec D = \rho &  \text{Gaußsches Gesetz} \\
\div \vec B = 0 & \text{Quellenfreiheit des B-Feldes} \\
\rot \vec H = \vec j + \frac{\partial \vec D}{\partial t} & \text{Ampersches Durchflutungsgesetz} \\
\rot \vec E = - \frac{\partial \vec B}{\partial t} & \text{Induktionsgestz}
\end{eqnarray}
Materialgleichungen: 
\begin{eqnarray}
\vec D = \epsilon \vec E \\
\vec B = \mu \vec H \\
\vec j = \sigma \vec E \\
\end{eqnarray}
Poynting-Vektor:
\begin{eqnarray}
\vec s = \vec E \times \vec H
\end{eqnarray}

Potential stromdurchflossener Leiter:

\begin{eqnarray}
\Phi (\vec r) = - \frac{Q}{2 \pi \epsilon l} \ln (\frac{r}{r_0}) + C \\
\vec E = - \nabla \Phi
\end{eqnarray}

Elektromagnetisches Vektorpotential: $\vec B = \rot \vec A$ \\
Skalares magnetisches Potential $\Phi$: $\vec E = - \nabla \Phi - \frac{\partial \vec A}{\partial t}$
\\
Eichfreiheit
\begin{eqnarray}
\vec A' = \vec A - \nabla \chi & \Phi'_\chi = \Phi + \frac{\partial \chi}{\partial t} \\
A = \frac{Vs}{m} & \Phi = V
\end{eqnarray}
Lorentz-Eichung: $\div \vec A + \epsilon M \frac{\partial \Phi}{\partial t} = 0$ \\
Coulomb-Eichung:  $\div \vec A = 0$

Zeitlicher Mittelwert: $ \overline{x} = \frac{1}{T} \int \limits_0^T x dt$
\end{document}